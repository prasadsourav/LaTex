% This is a comment, this will be not be read by the machine

% Which type of document do you want?
\documentclass[a4paper,12pt]{article} 

% Use packages for different functionalities
\usepackage[utf8]{inputenc} % For character encoding
\usepackage{natbib} % For bibliography
\usepackage{amsmath} % For mathematics
\usepackage{amssymb}
\usepackage{graphicx} % For graphics
\usepackage{url} % For including web links
\usepackage{tikz} % For advanced graphics with the language TikZ
\usepackage[font=footnotesize]{caption} 
\linespread{1.33} % Control spread between lines
\usepackage[toc,page]{appendix} 
\usepackage[colorlinks=true,linkcolor=red,urlcolor=red,citecolor=blue,backref=page]{hyperref} 
\usepackage{rotating}
\usepackage{amsthm}
\usepackage{lipsum}
% \usepackage[top=2.5cm, bottom=2.5cm, left=3cm, right=3cm]{geometry}

% Now comes the main part
\title{Introduction to \LaTeX}
\author{Sourav Prasad}
%\date{} %comment this out to include date

\begin{document}

\maketitle

\section{first section }

This is the first section of your project\footnote{this is a footnote}.This includes another footnote\footnote{here it is. see the magic}\
The rest texts means nothing and randomly generated texts in \LaTeX to show you how to write. With \textbf{lipsum} package we can just fill the page to see the formatting.\


\lipsum[]
\subsection{subsection 1}
\lipsum[]
\subsubsection{subsubsection 1}
\lipsum[]

\section{section 2}
\lipsum[]
\subsection{subsection 1}
\lipsum[]
\subsection{subsection 2}
\lipsum[]
\subsubsection{subsubsection 1}
\lipsum[]
\subsubsection{subsubsection 2}
\lipsum[]
\subsubsection{subsubsection 3}
\lipsum[]
\end{document}
